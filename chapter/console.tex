%! Author = paulsen
%! Date = 12.09.23

\begin{frame}{Die Konsole}
    \section{Die Konsole}\label{sec:die-konsole}
\end{frame}

\begin{frame}{Die "Wurzel"}

    \begin{quote}
        Die Wurzel (/) ist der Ursprung aller Dinge in Linux.
    \end{quote}

    \begin{itemize}
        \item Wurzel ist ähnlich zur "C:\textbackslash"-Festplatte in Windows
        \item In "/home" leben alle Nutzer und ihre Daten
        \item Dateisystem beginnt hier
    \end{itemize}

\end{frame}

\begin{frame}{Das Dateisystem}
    \subsection{Dateisystem}\label{subsec:dateisystem}

    \begin{quote}
        Alles in Linux ist eine Datei!?
    \end{quote}

    \begin{itemize}
        \item Konfigurationen (/etc)
        \item Commands (/bin)
        \item Geräte (/dev)
        \item Speichermedien (/media /mnt)
    \end{itemize}

\end{frame}

\begin{frame}{Die Shell}
    \subsection{Die Shell}\label{subsec:die-shell}

    Die Shell ermöglicht direkten Zugriff auf das Betriebssystem.

    \begin{itemize}
        \item Das mächtigste Werkzeug in Linux
        \item Navigation durch das Dateisystem
        \item Ausführen von System-Befehlen
        \item Anzeige von Informationen
    \end{itemize}

    \vspace{0.5cm}
    \includegraphics[width=3cm]{plasma-konsole}

    \vspace{0.5cm}
    \begin{alertblock}{Aufgabe}
        Öffnet die Konsole und führt "whoami" aus.
    \end{alertblock}

\end{frame}

\begin{frame}{Befehle}
    \subsection{Befehle}\label{subsec:befehle}

    Ein Befehl besteht aus bis zu drei Teilen

    \begin{enumerate}
        \item Befehlsname
        \item Optionen
        \item Argumente
    \end{enumerate}

    \vspace{0.5cm}
    \begin{exampleblock}{Beispiel}
        \$ ls -la /home/Nutzer/Dokumente
    \end{exampleblock}

    \vspace{0.5cm}
    \begin{alertblock}{Aufgabe}
        Probiert diesen Befehl an eurem Rechner mit und ohne den Optionen bzw Argumenten aus.
    \end{alertblock}

\end{frame}

\begin{frame}{Navigation}
    \subsection{Navigation}\label{subsec:navigation}

    Wie navigiere ich durch das Dateisystem?

    \textrightarrow "cd" wechselt den aktuellen Ordner

    \begin{itemize}
        \item[\$] cd Ordnername
        \item[\$] cd ..
        \item[\$] cd
    \end{itemize}

    \vspace{0.5cm}
    \begin{alertblock}{Aufgabe}
        \begin{itemize}
            \item Probiert die oben genannten Befehle aus.
            \item Navigiert zum Ordner mit den Kurs-Dokumenten.
        \end{itemize}
    \end{alertblock}

\end{frame}

\begin{frame}{Befehlshilfe}
    \subsection{Befehlshilfe}\label{subsec:befehlshilfe}

    Hilfe ich kenne diesen Befehl nicht!\pause

    \textrightarrow Zur Hilfe für unbekannte Befehle gibt es "man".

    \vspace{0.5cm}
    \begin{alertblock}{Aufgabe}
        Ausprobieren:

        \begin{itemize}
            \item[\$] man man
            \item[\$] man ls
            \item[\$] man
        \end{itemize}
    \end{alertblock}

    Wie komme ich da jetzt raus?

    \textrightarrow Q drücken

\end{frame}

\begin{frame}{Textbearbeitung}
    \subsection{Textbearbeitung}\label{subsec:textbearbeitung}

    "nano" ist ein CLI-Programm zum Bearbeiten und Erstellen von Dateien.

    \begin{itemize}
        \item CTRL + X zum Beenden
        \item CTRL + O zum Speichern
        \item CTRL + C zum Abbrechen des Speicherprozesses
    \end{itemize}

    \vspace{0.5cm}
    \begin{alertblock}{Aufgabe}
        \begin{itemize}
            \item Erstellt eine Datei mit nano
            \item[\$] nano test.txt
        \end{itemize}
    \end{alertblock}

\end{frame}

\begin{frame}{Dateien}
    \subsection{Dateien}\label{subsec:dateien}

    Umgang mit Dateien:

    \begin{itemize}
        \item Bearbeiten: \$ nano datei.txt
        \item Inhalt: \$ cat datei.txt
        \item Entfernen: \$ rm datei.txt
        \item Kopieren: \$ cp datei.txt neu.txt
        \item Verschieben: \$ mv datei.txt neu.txt
    \end{itemize}

    \vspace{0.5cm}
    \begin{exampleblock}{Tipp}
        Der "man"-Befehl kann beim Verständnis helfen.
    \end{exampleblock}

\end{frame}


\begin{frame}{Dateien}

    \begin{alertblock}{Aufgabe}
        \begin{enumerate}
            \item TODO
            \item TODO 2
        \end{enumerate}
    \end{alertblock}

\end{frame}

\begin{frame}{APT}
    \subsection{APT}\label{subsec:apt}

    APT ist der wichtigste Paket-Manager auf Debian/Ubuntu Systemen.



\end{frame}