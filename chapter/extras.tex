%! Author = paulsenik
%! Date = 01.12.23

\begin{frame}{Extras}
    \section{Extras}\label{sec:extras}

    Ein paar Experten-Aufgaben für die Schnellen.

    Du wirst für manche Aufgaben Hilfe aus dem Internet brauchen.

\end{frame}

\begin{frame}{Extras - KDE Plasma}
    \subsection{KDE Plasma}\label{subsec:kde plasma}

    \begin{alertblock}{Aufgaben}
        \begin{enumerate}
            \item Füge eine neue Schriftart zum System hinzu
            \item Komprimiere einen Ordner als Zip-Datei
            \item Erstelle einen Screenshot und speichere diesen ab
            \item Erstelle einen Autostart für Thunderbird
            \item Erstelle einen Desktop-Shortcut für Thunderbird
            \item Installiere einen neuen Mauszeiger (Cursor)
        \end{enumerate}

    \end{alertblock}

\end{frame}


\begin{frame}{Extras - Software}
    \subsection{Software}\label{subsec:software}

    \begin{alertblock}{Aufgaben}
        Installiere:
        \begin{enumerate}
            \item "Flatpak"
            \item Die "Flatpak-Discover" integration
            \item "KColorChooser" mithilfe von Flatpak
            \item Ein AppImage aus dem Internet
            \item Ein OpenSource Programm deiner Wahl aus dem Web
            \item "Bottles", für Windows-Programme
            \item Installiere ein Windowsprogramm deiner Wahl mithilfe von "Bottles".
        \end{enumerate}

    \end{alertblock}

\end{frame}


\begin{frame}{Extras - Konsole}
    \subsection{Shell}\label{subsec:shell}

    Jede Aufgabe ist in der Konsole machbar.

    \vspace{0.5cm}
    \begin{alertblock}{Aufgaben}
        \begin{enumerate}
            \item Ändere deinen Rechnernamen
            \item Erstelle einen neuen Nutzer und füge ihn zu der Gruppe "sudo" hinzu.
            \item Kopiere mithilfe von Platzhaltern und Wildcards ("* ? [abc]") alle Dateien, die ein "a" im Namen haben, in einen anderen Ordner.
        \end{enumerate}
    \end{alertblock}

\end{frame}


\begin{frame}{Extras - Konsole 2}

    \begin{alertblock}{Aufgaben}
        Probiere und erkundige dich über folgende Befehle:
        \begin{enumerate}
            \item[\$] mkdir
            \item[\$] xkill
            \item[\$] htop
            \item[\$] grep
            \item[\$] chmod
            \item[\$] chown
        \end{enumerate}

    \end{alertblock}

\end{frame}
