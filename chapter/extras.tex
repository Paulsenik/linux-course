%! Author = paulsenik
%! Date = 01.12.23

\begin{frame}{Extras}
    \section{Extras}\label{sec:extras}
    \pause
    Für manche Aufgaben wirst du Hilfe aus dem Internet benötigen.

\end{frame}

\begin{frame}{Extras - KDE Plasma}
    \subsection{KDE Plasma}\label{subsec:kde plasma}

    \begin{alertblock}{Aufgaben}
        \begin{enumerate} \pause
            \item Füge eine neue Schriftart zum System hinzu\pause
            \item Komprimiere einen Ordner als Zip-Datei\pause
            \item Erstelle einen Screenshot und speichere diesen ab\pause
            \item Erstelle einen Autostart für Thunderbird\pause
            \item Erstelle einen Desktop-Shortcut für Thunderbird\pause
            \item Installiere einen neuen Mauszeiger (Cursor)
        \end{enumerate}

    \end{alertblock}

\end{frame}


\begin{frame}{Extras - Software}
    \subsection{Software}\label{subsec:software}

    \begin{alertblock}{Aufgaben}
        \pause
        Installiere:\pause
        \begin{enumerate}
            \item "Flatpak"\pause
            \item Die "Flatpak-Discover" integration\pause
            \item "KColorChooser" mithilfe von Flatpak\pause
            \item Ein AppImage aus dem Internet\pause
            \item Ein OpenSource Programm deiner Wahl aus dem Web\pause
            \item "Bottles", für Windows-Programme\pause
            \item Installiere ein Windowsprogramm deiner Wahl mithilfe von "Bottles".
        \end{enumerate}

    \end{alertblock}

\end{frame}


\begin{frame}{Extras - Konsole}
    \subsection{Shell}\label{subsec:shell}
    \pause
    Jede Aufgabe ist in der Konsole machbar.
    \pause
    \vspace{0.5cm}
    \begin{alertblock}{Aufgaben}
        \begin{enumerate}\pause
            \item Ändere deinen Rechnernamen\pause
            \item Erstelle einen neuen Nutzer und füge ihn zu der Gruppe "sudo" hinzu.\pause
            \item Kopiere mithilfe von Platzhaltern und Wildcards ("* ? [abc]") alle Dateien, die ein "a" im Namen haben, in einen anderen Ordner.
        \end{enumerate}
    \end{alertblock}

\end{frame}


\begin{frame}{Extras - Konsole 2}

    \begin{alertblock}{Aufgaben}
        \pause
        Probiere und erkundige dich über folgende Befehle:
        \pause
        \begin{enumerate}
            \item[\$] mkdir\pause
            \item[\$] xkill\pause
            \item[\$] htop\pause
            \item[\$] grep\pause
            \item[\$] chmod\pause
            \item[\$] chown
        \end{enumerate}

    \end{alertblock}

\end{frame}
