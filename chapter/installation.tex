%! Author = paulsen
%! Date = 12.09.23

\begin{frame}{Installation}
    \section{Installation}\label{sec:installation}
\end{frame}

\begin{frame}{Was ist Linux?}
    \subsection{Linux?}\label{subsec:linux?}

    \begin{quote}<1->
        Als GNU/Linux bezeichnet man in der Regel freie, unixähnliche Mehrbenutzer-Betriebssysteme, die auf dem Linux-Kernel und wesentlich auf GNU-Software basieren.
    \end{quote}

    \begin{itemize}
        \item<2-> 1991 als Alternative zu UNIX erschaffen
        \item<3-> Freie und offene Alternative zu Windows und MacOS
        \item<4-> Unterstützung von großen Unternehmen (Google, Microsoft, Facebook, etc.)
    \end{itemize}
    \begin{alertblock}<1->{Fun Fact}
        Linux ist das größte Softwareprojekt der Welt.
    \end{alertblock}

\end{frame}

\begin{frame}{Linux Distributionen}
    \subsection{Distributionen}\label{subsec:distributionen}

    Ein großteil der Distributionen (Sorten) von Linux ist Teil dieser 3 "Familien":

    \pause

    \begin{itemize}
        \item Arch
        \item Debian
        \item RHEL (Red Hat Enterprise Linux)
    \end{itemize}

\end{frame}

\begin{frame}{Praxis}
    \subsection{Praxis}\label{subsec:praxis}

    !TODO!

\end{frame}