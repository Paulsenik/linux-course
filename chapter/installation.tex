%! Author = paulsen
%! Date = 12.09.23

\begin{frame}{Installation}
    \section{Installation}\label{sec:installation}
\end{frame}

\begin{frame}{VirtualBox}
    \subsection{VirtualBox}\label{subsec:VirtualBox}
    Öffne Virtual Box und klicke auf "New".
    \includegraphics[width=10cm]{VirtualBox_1}
\end{frame}

\begin{frame}{VirtualBox}
    Gebe Namen und Installationsort ein.
    \includegraphics[width=10cm]{VirtualBox_2}
\end{frame}

\begin{frame}{VirtualBox}
    \includegraphics[width=10cm]{VirtualBox_3}
\end{frame}

\begin{frame}{VirtualBox}
    \includegraphics[width=10cm]{VirtualBox_4}
\end{frame}

\begin{frame}{VirtualBox}
    Wähle VDI aus
    \includegraphics[width=10cm]{VirtualBox_5}
\end{frame}

\begin{frame}{VirtualBox}
    \includegraphics[width=10cm]{VirtualBox_6}
\end{frame}

\begin{frame}{VirtualBox}
    Füge eine Virtuelle Festplatte mit 10-20GB hinzu.
    \includegraphics[width=10cm]{VirtualBox_7}
\end{frame}

\begin{frame}{VirtualBox}
    Gehe in den Einstellungen der VM auf "Storage" und wähle die ISO-Datei aus.
    \includegraphics[width=10cm]{VirtualBox_8}
\end{frame}

\begin{frame}{Setup}
    \subsection{Setup}\label{subsec:setup}

    !TODO!

    \begin{exampleblock}<1->{Fun Fact}
        Die 500 Schnellsten Supercomputer der Welt laufen auf Linux
    \end{exampleblock}

\end{frame}